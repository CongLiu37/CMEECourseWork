\documentclass[11pt]{article}

\usepackage{setspace}
\usepackage[colorlinks]{hyperref}
\usepackage{lineno}
\usepackage{booktabs}
\usepackage{graphicx}
\usepackage{float}
\usepackage{floatrow}
\usepackage{subfigure}
\usepackage{caption}
\usepackage{subcaption}
\usepackage{geometry}
\usepackage{multirow}
\usepackage{longtable}
\usepackage{lscape}
\usepackage{booktabs}


\title{Seminar Diary}

\author{Cong Liu\footnotemark[1]}

\renewcommand{\thefootnote}{\fnsymbol{footnote}}
\footnotetext[2]{Department of Life Sciences (Silwood Park), Imperial College of Science, Technology and Medicine}

\date{2021}

\linespread{1.5}
\geometry{left=2cm,right=2cm,top=2cm,bottom=2cm}
\begin{document}
  \maketitle
  \newpage
    
  \linenumbers
  \section{Sleeping in a society of honey bees: the tale of a sleep-deprived dancer and her unwitting followers, and the search of insect dreams}
  Lecturer: Barrett Klein, University of Wisconsin.
  \newline
  Date: 2020.10.15
  \newline
  Honey bee (\textit{Apis mellifer}) foragers share information on the destinations of resources (food, water, \textit{etc}) to nest mates via waggle dances. 
  In waggle dances, the direction one forager waggles relative to vertical direction indicates the direction of the food resource relative to the sun, while the time it waggles is closelty correlated with distance to the resource.
  \newline 
  With sleep deprivation, the preciseness of foragers' waggle dances is decreased. 
  Sleep-restricted bee dancers performed fewer dances compared with controls. 
  They also performed proportionally fewer precise dances, or precise dances with fewer waggle phases, either leads to decrease in information load of waggle dances. 
  As for nest mates following imprecise dances, they were more likely to switch dances rather than exit, following fewer waggling phases. 
  These findings indicate that sleep loss is likely to impact negatively on both delivery and receiving of information vai waggle dances of honey bees. 
  They are of consensus with previous studies revealing the role of sleep in communication and learning consolidation. 
  \newline
  Another interesting topic is whetehr honey bees dream during sleep. 
  In order to examine it, the first thing need to be seen is what is happening in the honey bee brain during sleep. 
  By imaging brain activity of sensing odors of honey bees while they sleep, disperse activities were found, indicating the possibility of dreaming.
  \newline
  In summary, sleep deprivation impacted the communication between honey bees negatively, and disperse activities during sleep were found in sensing odors of honey bees.
  
  \newpage
  \section{Thermal ecology of infectious disease}
  Lecturer: Martha Socket, University of Califormia Los Angeles
  \newline
  Date: 2020.10.22
  \newline
  Temperature is an important abiotic factor that influences ecological processes including infectious disease transmission, and research on it has been motivated by climate change. 
  Investigations in mosquito-borne viruses and a fungal parasite (\textit{Metschnikowia bicuspidata}) infecting freshwater zooplankton (\textit{Daphnia dentifera}) helped understand the thermal response of pathogen transmission, which is often nolinear and combining multipe mechanisms. 
  \newline 
  The analysis of 15 mosquito-pathogen system (10 pathogens and 9 vector species) showed unimodal response to temperature, while the effect of same temperature on transmission of different pathogens varies strongly. 
  Compared with tropical pathogens, temperate pathogens systematically have broader thermal ranges, lower thermal minima and optima. 
  This variation is determined by both vector and pathogen species, and restriction to transmission is provided by different traits.  
  Warming is expected to increase transmission under thermal optima, but decrease transmission above thermal optima. 
  \newline
  Investigation in \textit{D. dentifera} and fungal parasite \textit{M. bicuspidata} illustrated a novel mechanism through which temperature affects transmission. 
  Warm temperatures lead to high transmission via impacting host feeding rate and parasite spore development, shaping epidemic size and seasonality.
  \newline
  In summary, these investigations highlight the impact of temperature on transmission of infectious diseases.

  \newpage
  \section{Eco-evo dynamics of experimentally tractable microbial systems}
  Lecturer: Jose I. Jimenez, Imperial College London
  \newline
  Date: 2020.10.29
  \newline
  Predictable manipulation of microbial community is valuable for multiple applications, but hindered by the complexity of interactions between microorganisms, as well as by interactions between microbial community and the environment. 
  Experimental population dynamics under lab-controlled conditions can provide insight into these interactions and potentiality for artificial modification of microbial community.
  \newline
  Investigations in horizonal transmission of antimicrobial resistance genes, which is mediated by F plasmid conjugation in \textit{Escherichia coli}, highlighted the influence imposed by population structure.  
  In liquid culture, conjugation can be inhibited by suppression of the formation of mating pairs using filamentous phages or M13 minor coat protein g3p. 
  In surface-associated growth, conjugation can be suppressed by severe genetic drift, which leads to spatial isolation of donor and recipient cells and thus restricts conjugation. 
  These results provide evidence that spatial structure of population could impose impact on the horizonal transfer of antimicrobial resistance genes.
  \newline
  Beside spatial structure, interventions of biological traits might be sufficient to drive population dynamics. 
  In \textit{Pseudomas aeruginosa}, mutation in a secondary receptor leads to decrease in the production of siderophore. 
  The effect of this mutation is dependent on the interactions with the environment. 
  It brings increase in fitness compared with wild-type, when high cost is required for the production of siderophore.
  \newline
  In summary, experimental microbial populations illustrate the impact of spatial structure and traits on population dynamics.

  \newpage
  \section{The honeybee waggle dance: evolutionary marvel but modern-day relic?}
  Lecturer: Elli Leadbeater, Royal Holloway University
  \newline
  Date: 2020.11.26
  \newline
  Honey bees \textit{Apis mellifera} share information on spatial location of resources with nest mates via waggle dances. 
  In order to investigate when spatial information from waggle dances is actually used by honey bees, network-based diffusion analysis was used to quantify the information flow and assess the impact of dance communication and hive-based olfactory information transfer on honey bee recruitment events. 
  The results show that successful recruits to novel locations are dependent on information from dances instead of olfactory cues, although the latter can also guide honey bees to the same locations. 
  However, in recruits to known locations, the importance of dances is reduced. 
  \newline
  Honey bee foragers recruit nest mates to novel locations with resources like food and water. 
  How do foragers evaluate the value of locations to make decision on recruiting nest mates? 
  To investigate in this question, decoding dances were used to quantify the forage value of landscapes for social bees. 
  By studying behaviour of honey bees in urban and rural areas, it is found that bees in urban area evaluate different landscapes quite evenly, while bees in rural area evaluate landscapes with more variances, and high values are assigned to oilseed rape.  

  \newpage
  \section{Contemporary evolution and adaptive divergence in open ocean environments: new insights and applications to fisheries mangement}
  lecturer: Nina Overgaard Therkildsen, Cornell University
  \newline
  Date: 2021.2.25
  \newline
  Pressures from human activities is increasingly threatenning the persistence of marine species, urging better understanding in how and how quick orginasms can genetically adapt to changing environmental conditions. 
  DNA sequencing methods enables effective genetic studies of non-model species at population level, providing possibility for investigations in local adaption of marine species and further applications on fishery management.
  \newline
  In Atlantic silverside \textit{Menidia menidia}, strong adaptive divergences were found despite high gene flow. 
  Dramatic genomic footprints of selection were found, corresponding to cryptic adaptive divergence. 
  Massive linkage disequilibrium blocks are fixed for opposite alleles despite of small divergence at genome level. 
  This may be caused by high selective pressure that overcomes gene flow between populations of different locations. 
  An alternative cause is chromosomal rearrangement, which potentially link locally adaptive alleles as "supergenes".
  \newline
  Such insight in marine population genetics can be important for fishery management. 
  It is found that size-selective fishing accerlates loss of genetic diversity. 
  Under selection, highly polygenic parallel allele frequency shifts. 
  Interestingly, under parallel phynotypic responses to selection, there can be idiosyncratic genomic responses. 

  \newpage
  \section{Semi-automated modeling of COVID-19 in the United States}
  Lecturer: John Drake, University of Georgia
  \newline
  Date: 2021.3.4
  \newline
  Modeling the spread of COVID-19 is helpful for understanding the transmission of epidemic and public policy making. 
  Such models are often characterized by high complexity in model structure, requiring great care to ensure they are fitted intelligently with biologically realistic coefficients, since good fit does not necessarily result from plausible parameters. 
  Initializing parameter search with previous well-fitted parameters can be a helpful strategy.  
  Also, assumptions should be continuously questioned as new information is gathered.
  \newline
  By modeling the transmission of COVID-19 in United States, it is revealed that increased social distance and decreased human mobility could significantly reduce the impact of pandemic. 
  Many places in United States have returned to normal conditions, indicating there is still room for improvement in controlling pandemic. 
  The model also illustrates the importance of early intervention, which is critical to controlling the pandemic. 
  \newline 
  Accoding to the results of the model, it may be possible in some states to release travelling restrictions without risking a resurgence in the near term. 
  However, this may be contingent on continuing other measures that reduce transmission. 


  \newpage
  \section{The evolution of human commensalism in Passer sparrows}
  Lecturer: Mark Ravinet, University of Nottingham
  \newline
  Date: 2021.3.11
  \newline
  Human commensalism is widespread. Many species benefit from human activities. 
  Passer sparrows are used to investigate what factors drive the evolution of human commensalism. 
  \newline
  Genomic analysis of house sparrows provides insight into the evolution of human commensalism. 
  The model shows that hoouse commensalism arose in Near East and divergented into two clades: one spreaded into south Europe and experienced extensive admixture events, leading to hybrid speciation of Italian sparrow; the other one spreaded further into Europe with agriculture. 
  \newline
  The evolution of human commensalism also causes phenotypic consequences, including sedentary, nesting in human structures and feeding on human resources. 
  It also impacts morphological traits including skull and beak shape, which are recognized as potential adaptive traits to clutivated cereals. 
  \newline 
  Human commensalism also leads to strong divergent selection between house and Bactrianus sparrows. 
  On chromosome 8, two genes are strongly selected. 
  One is \textit{COL11A1}, whose homologies in human are related with Marshall syndrome. 
  The other one is associated with starch digestion in humans and dogs. 
  These findings are likely to be helpful for understanding functional shifts accompanied with origin of human commensalism.

  \newpage
  \section{Eco-evolutionary dynamics of coronaviruses}
  Lecturer: Jessica Metcalf, Princeton University
  \newline
  Date: 2021.5.6
  \newline
  Coronaviruses represent a vast, diverse and zoonotic reservoir of viruses, although there are only four phylogenetically distant endemic coronaviruses discovered in human in past two decades. 
  The successful emergence of a endemic pathogen is dependent on success in threading the needle in terms of virulence and transmission, and in evading existing immunity in host populations. 
  Beyond ascertainment and change in recent conditions, the intersection between transmission, virulence and the existing immune protection is likely to be the answer for why there is only a small number of endemic coronaviruses.
  \newline
  Facing the imapct of SARS-CoV-2, vaccines show clear benefits vastly outweigh risks, although they may drive the evolution of viruses. 
  Vaccines have theoretical potentiality to select viruses with increased virulence if they fail to provide protection, which would likely to cause cost falling on people without access to vaccines. 
  The evoluton of virulence should be minotored in preparation for further impact on human society.
  \newline
  Still few understanding in reinfection of coronaviruses is available, although it is important for understanding the effect of vaccination. 
  Burden and timing of SARS-CoV-2 infections are shaped by immune responses and vaccine doses in both short and long term. 
  One-dose-strategy might be good in short term in order to increase the number of immunized people, but in longer term, two-dose-strategy could mitigate the possibility of antigenic evolution.

  \newpage
  \section{From 1 ul of DNA to ecosystem scale biodiversity in four easy steps: the fantastic world of eDNA analysis}
  Lecturer: Si Creer, Bangor University
  \newline
  Date: 2021.5.20
  \newline
  Environmental DNA (eDNA) refers to free DNA molucules in the environment, released from multiple species. 
  It provides potentiality for efficient and effective assessment of biodiversity. 
  Multiple methods can be used for manipulation of eDNA, including quantative polymerase chain reaction, metabarcoding and shotgun eDNA sequencing. 
  The following case studies illustrate the application of eDNA in investigations in biodiversity.
  \newline
  The first case study focuses on diversity of pollen in the air and its fluctuation in time and space. 
  The results show that aerial grass pollen communities shift in beta composition throughout the grass flowering season. 
  The second case studies look into soil samples. 
  It is found that land uses imposes impact on richness of different communities. 
  After sorting land usage types in the productivity gradient, the richness of bacterial, fungal and protist communities follow identical types of pattern, while archaea and animal communities behave in different ways. 
  Using the same dataset, the relationships between textural heterogeneity and other factors are illustrated. 
  Results show that bacterial community is related with textural heterogeneity, but fungal community is not. 
  The third case is analysis of freshwater samples from a lake ecosystem, assessing annual dynamics of community. 
  The last case assesses foraging of honey bee via analysis eDNA extracted from honey, revealing historical changes in floral resources.

  \newpage
  \section{Living together: cooperation in the sociable weaver}
  Lecturer: Rita Covas, University of Porto
  \newline
  Date: 2021.5.27
  \newline
  Cooperation is widespread in nature and diverse in fuctions. 
  It brings benefits to the group, but costly to the individual. 
  To understand the cooperation and conflict in groups of organisms, sociable weaver is a suitable study model. 
  They cooperate in four distinct behaviours: nest building, helping at the nest, predator mobbing and vigilance. 
  \newline
  In cooperation of breeding, helpers of sociable waevers do not reproduce, but help breed offsprings of other group members. 
  A possible reason for the delay of breeding in helper sociable waevers is that breeding under poor conditions is too costly. 
  This hypothesis is verified by increase in independent breeding and decrease in helping after artificial food supplementation. 
  \newline
  To understand the cost of helping, free radical activity was measured under experimental increase in the cost of helping. 
  As a result, in young individuals, probability of helping increases if they are in poor physiolohical conditions. 
  Experimental increase in the energy cost of helping leading to increase in free radical damage and decrease in feeding rate. 
  \newline
  In summary, there is a trade-off between investment in helping and self maintenance. Individuals may adjust their behaviour in order to minimise possible costs.
  \newline
  It is also found that most helpers are related to the breeders. 
  Cooperation in breeding is generally positive on its success and allows breeders to decrease their efforts in caring for brood. 
  It also improved the suvival of female breeders.

  \newpage
  \section{Tracing the evolutionary history of viruses from the Stone Age to the present using ancient DNA}
  Lecturer: Martin Sikora, University of Copenhagen
  \newline
  Date: 2021.6.10
  \newline
  The recovery of ancient pathogen DNA provides valuable insights into the origin epidemics in human history, especially those lacking reliable historical records. It also provides a direct window into the evolution of pathogens, generating understanding in the general principles of evolutionary biology.
  \newline 
  From ancient human remains, genomes of variola viruses (samllpox) were recovered, revealing that a previously unknown clade was widespread in Northern Europe during the Viking Age. This clade of variola viruses from Viking Age and the viruses spreaded in 19th century evolved from a common ancestor at approximate 1700 years ago, via distinct paths of gene inactivation.
  \newline
  Another case of investigation in the genomes of ancient pathogens looks into human adenovirus with a history of 3100 years. 
  It shows that adenovirus that causing common childhood infections today were already spreading in human communities in Upper Palaeolothic. The genome of this virus shows low rate of molecular evolution, indicating its long-term co-divergent with host. The identification of ancient intra-type divergences and recent within-type variations indicate that it is possible that adenovirus was capable of human transmission in ancient time.

  \newpage
  \section{UK's work on the global food system: how we use research to further our work with government and industry}
  Lecturer: Mollie Gupta, WWF
  \newline
  Date: 2021.5.13
  \newline
  Food is the greatest part in ecological foodprint of humanity. 
  The food system is one of major drivers of biodiversity loss and greenhouse gas emissions. 
  Although a huge amount of land has been used for food production, yet there are regions where indernourishment is still a challenge, while other regions where there is overconsumption.
  \newline
  There are several drivers making food system consuming a huge amount of resources, one of them is overconsumption of animal protein. 
  Besides, the production of animal protein is inefficient, and a third of agricultural production is wasted. 
  Under such conditions, there are things need to happen including cut food loss and waste, increase food productivity through nature positive approaches and adoption of sustainable diet choices. 
  Furthermore, improvenment in goverance, planning and use of landscapes, including restoration of degraded ecosystems, is needed. 
  \newline 
  Facing these on-going issues, UK government set up policies. 
  A mandatory due diligence obligation is introduced on companies that place commodities and derived products that contribute to deforestation. Actions are taken to ensure similar principles are applied to the finance industry.
   


  



  

  




%   \newpage
%   \section{Ecosystem feedbacks to climate change: when and where do microbes matter?}
%   Lecturer: Bonnie Waring, Grantham Institute on Climate Change/DoLS
%   Date: 2020.11.12
%   Carbon dioxide accumulation in the atomosphere causes climate change, and soil organic matter represnts the largest terrestrial carbon pool, whose dynamics leads to uncertaintity in predictions of feedback on climate change. 
%   The cycle of soil carbon is governed by microbial decomposers, which may have impact on ecosystem responses to climate change in multiple aspects.
%   \newline
%   Investigations in tropical dry forest highlight the role of soil fungal community in plant growth. 

%   When the soil is nutritionally poor, the plant community growing on the soil has high values of nutrient response efficiency, \textit{i.e.} high productivity per unit soil nutrient. 
%   No correlation between nitrogen response efficiency and phosphorus response efficiency, indicating they are controlled by different traits. 
%   This is consistent with prior understanding in nutrition absorbing, which is mediated by fungal symbionts. 



%   Decomposition rates were correlated with tree and soil fungal community composition as well as soil fertility, but these relationships differed among litter types. In low fertility soils dominated by ectomycorrhizal oak trees, bulk litter turnover rates were low, regardless of soil moisture. By contrast, in higher fertility soils that supported mostly arbuscular mycorrhizal trees, bulk litter decay rates were strongly dependent on seasonal water availability. Both measures of decomposition increased with forest age, as did the frequency of termite-mediated wood decay. Taken together, our results demonstrate that soils and forest age exert strong control over decomposition dynamics in these tropical dry forests, either directly through effects on microclimate and nutrients, or indirectly by affecting tree and microbial community composition and traits, such as litter quality.
   
%   No correlation between nitrogen response efficiency and phosphorus response efficiency, indicating they are controlled by different traits. 
%   Soil fungal communities respond to soil nutrients.


  \end{document}