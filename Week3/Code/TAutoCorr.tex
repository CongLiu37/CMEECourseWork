%Language: LaTeX
%Author: Cong Liu (cong.liu20@imperial.ac.uk)
%Script: TAutoCOrr.tex
%Work directory: CMEECourseWork/Week3/Code
%Description: Written work of practical "Autocorrelation in weather"
%PS. Figure 1 is given by TAutoCorr.R in pdf format. When running, the figure should be in Code

\documentclass[12pt]{article}

\title{Autocorrelation of Annual Temperature in Key West, Florida for the 20th Century}

\author{Cong Liu (cong.liu20@imperial.ac.uk)}

\date{Oct, 2020}

\usepackage{graphicx}
\usepackage{float}
\usepackage{subfigure}
\usepackage[colorlinks]{hyperref}
\usepackage{hyperref}

\begin{document}
  \maketitle

  \begin{abstract}
    This study concerns whether temperatures of years are autocorrelated. Temperatures
    in Key West, Florida for 20th century were used. Lag 1 autocorrelation was tested, and 
    \textit{p}-value was estimated using a simulation approach. The results showed a weak but significant
    correlation between temperature of one year and its successive year. 
    
  \end{abstract}
  
  \section{Introduction}
  Autocorrelation refers to correlation between current values of a variable and its past values.
  If autocorrelation is found between values that are \textit{k} time periods apart, it is called
  a lag \textit{k} autocorrelation. In this paper, autocorrelation in temperatures of Key West, 
  Florida in 20th century was tested and \textit{p}-value was estimated, showing a weak but significant
  lag 1 autocorrelation.
  
  \section{Materials \& Methods}
   
    \subsection{Data of Temperatures}
    The raw data file of temperatures is named as KeyWestAnnualMeanTemperature.Rdata, 
    which is accessible in 
    \newline
    \href{https://github.com/mhasoba/TheMulQuaBio/tree/master/content/data}{https://github.com/mhasoba/TheMulQuaBio/tree/master/content/data}.
    It was plotted by a R script.
    
    \subsection{Autocorrelation Test}
    A R script was used to calculate Pearson's correlation coefficient between temperatures that
    are 1 year apart.
    
    \subsection{Estimation of \texorpdfstring{$\mathit{p}$}{}-value}
    A R script is used to estimate \textit{p}-value. To begin with, values of temperature were
    rearranged randomly, and Pearson's correlation coefficient of lag 1 autocorrelation was calculated.
    This process was repeated 10,000 times, and the fraction of correlation coefficients whose absolute 
    values are larger than that from raw data is approximate \textit{p}-value.
  
  \section{Results}
    
    \subsection{Plot of Raw Data}
    Figure 1 illustrates temperatures in Key West, Florida for 20th century.
    \begin{figure}[H]
        \centering
        \includegraphics[width=0.7\textwidth]{TAutoCorr_Figure1.pdf}
        \caption{Temperature in Key West, Florida}
        \label{Fig. 2}    
    \end{figure}

    \subsection{Test of Autocorrelation}
    The correlation coefficient of temperatures 1 year apart is 0.3262, and approximate 
    \textit{p}-value was 0.0013.

  \section{Discussion}
  Correlation coefficient measures the extent that two variables are linearly correlated.
  A positive correlation coefficient means positive correlation between two variables. If 
  the correlation coefficient is negative, then there is a negative correlation. The closer
  the absolute value of correlation coefficient to 1, the stronger the correlation is. Here, the
  correlation coefficient between one-year-apart temperatures is 0.3262, showing that there is a weak 
  positive correlation between the temperatures of one year and its successive year.
  \newline
  The statistical significance is measured by \textit{p}-value. It is defined as the probability that
  null hypothesis is rejected while it is true, namely, the false positive probability. A 
  small \textit{p}-value means high significance. Here, the 
  estimated \textit{p}-value was 0.0013, indicating the correlation is significant.

  \section{Conclusion}
  Overall, a weak but significant lag 1 autocorrelation was found in temperatures of Key West, Florida 
  for 20th century.

  \section{Supplement}
  All codes are wraped in TAutoCorr.R, which is accessible in 
  \newline
  \href{https://github.com/CongLiu37/CMEECourseWork/tree/main/Week3/Code}{https://github.com/CongLiu37/CMEECourseWork/tree/main/Week3/Code}
\end{document}
